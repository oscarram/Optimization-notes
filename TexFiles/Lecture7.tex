\begin{remark}
Some elementary properties of conjugate functions
\begin{itemize}
	\item  \textbf{Young inequality} $J(u)+J^*(p^*)\geq p^*(u) \quad \forall u \in U, \forall p^* \in U^*$
	\item  $J^*(0)=\sup_{u\in U}\left((0,u)-J(u)\right)=\sup_{u\in U}\left(-J(u)\right)=\inf_{u\in U}J(u)$
\end{itemize}
In many applications in optimization, is used the equivalent formulation,
\[
\inf_{u\in U}J(u)=-J^*(0).
\]
\end{remark}
$J\leq F \implies J^*\geq F^*$

\begin{theorem}
	Let $U$ a Banach space and $J^*: U^* \rightarrow \overline{\mathbb{R}}$ be the conjugate of the $J:U\rightarrow \overline{\mathbb{R}}$. Then for all $u\in U$. \[p^* \in \partial J(u) \iff J(u)+J^*(p^*)=p^*(u)\]. \label{th7: conjugate plus functional}
	\begin{proof}
		content...
	\end{proof}
\end{theorem}
\begin{corollary}
	It follows from previous theorem that $\partial J(u)=\lbrace p^* \in U^* | J(u)+J^*(p^*)=(p^*, u)\rbrace$.
\end{corollary}
\begin{theorem}
	Let $U$ be a Banach space and $J:U\rightarrow \mathbb{R}$ be proper function. If $p^* \in \partial J(u)$ then $u\in \partial J^* (p^*)$
	\begin{proof}
		Let $p^*\in \partial J(u)$. For any $g^* \in U^*$, it follows \[J^*(g^*)=\sup_{v\in U} \left\lbrace
		g^*(v)-J(v)\right\rbrace\geq g^*(u)-J(u)\geq g^*(u)-J(u)\]
		From theorem \ref{th7: conjugate plus functional} 
		\[
		J^*(g^*)\leq g^*(u)-p^*(u)+J^*(p^*) = \left(g^* - p^* \right)(u)+J^*(p^*)\implies u\in \partial J^*(p^*).
		\]
	\end{proof}
\end{theorem}
By iteration the definition, we obtain the bipolar function
$(J^*)^*=J^{**}: U^{**}\rightarrow \overline{\mathbb{R}}$, 
\[
	J^{**}(u)=\sup_{p^*\in U^*} \lbrace p^*(u)-J^*(p^*) \rbrace
\]

\begin{theorem}
	Let $U$ be a reflexive Banach space. The $J^{**}$ is the maximum convex functional below $J$ (also called convex envelope), i.e. $J^{**}(u)\leq J(u)$, $\forall u \in U$ and $F(u)\leq J^{**}(u)$, $\forall u \in U$ if $F$ is also convex and $F(u)\leq J(u)$, $\forall u$. In particular $J^{**}=J$ if and only if $J$ is convex.
	\begin{proof}
		\begin{align}
			J^{**}(u)&=\sup_{p^*\in U^*} \left\lbrace p^*(u)-J^*(p^*) \right\rbrace\\
					&=\sup_{p^*\in U^*} \left\lbrace p^*(u)-\sup_{v\in U}\left\lbrace p^*(v)-J(v)\right\rbrace\right\rbrace\\
					&=\sup_{p^*\in U^*} \left\lbrace p^*(u)+\inf_{v\in U}\left\lbrace p^*(v)-J(v)\right\rbrace\right\rbrace\\
		\end{align}
		Since for any $p^* \in U^*$,
			\[
				\inf_{v\in U}\left\lbrace p^* \left(u-v\right)+J(v) \right\rbrace \leq p^*(u-u)+J(u)
			\]
		We have that $J^{**}(u)\leq J(u)$.
		
		Now we assume that $F$ is a convex functional and $g^* \in \partial F(u)$ for $u \in U$.
		\begin{align}
			\implies  F(v)&\geq F(v)+q^*(v-u)\\
					  F^{**}(u)& = \sup_{p^* \in U^*} \inf_{v \in U} \left\lbrace p^*(u-v)+F(u)+q^*(v-u)\right\rbrace \\
					  &\geq \sup_{p^* \in U^*} \inf_{v \in U} \left\lbrace \left(p^*-q^*\right)(u-v)+F(u)\right\rbrace\\
					  &\geq \inf_{v \in U} \left\lbrace (q^*-q^*)(u-v)+F(u) \right\rbrace\\
					  &=F(u)
		\end{align}
	If $F$ is convex,
	\begin{equation}
		\implies F(u)\leq F^{**}(u)\leq F(u) \implies F(u)=F^{**}(u),
	\end{equation}

	\begin{equation}
	F(u)=F^{**}(u)=\sup_{p^* \in U^*} \inf_{v \in U} \left\lbrace p^*(u-v)+F(v)\right\rbrace \leq J^{**}(u)
	\end{equation}
	\end{proof}
\end{theorem}
