We check the optimality conditions.
\begin{align*}
	0&=\Phi (\overline{u},0)+\Phi^*(0, \overline{p^*})\\
	&= F(\overline{u})+G(A\overline{u})+F^*(A^\star\overline{p^*}) +G^*(-\overline{p^*})\\
	&= [F(\overline{u})+F^*(A^\star\overline{p^*})-A^\star\circ p^*(\overline{u})]+[G(A\overline{u})+G^*(-\overline{p^*})-(-p^*)(A\overline{u})]
\end{align*}

Using Young inequality $J(u)+J^*(u^*)-u^*(u)\geq 0$, $\forall u \in U$, and $\forall u^* \in U^*$, we see that both square brackets are nonnegative; and the sum is zero. Then
\begin{align*}
	F(\overline{u})+F^*(A^\star\overline{p^*})=A^\star\circ p^*(u) &\implies A^\star p^* \in \partial F(\overline{u}) \\
	G(A\overline{u})+G^*(-\overline{p^*})=(-p^*)(A\overline{u}) &\implies -p^* \in \partial G(A\overline{u})
\end{align*}

$F$, $G$ are convex and locally bounded, one can show that 
$\sup\eqref{eq8. P*}=\inf\eqref{eq8. P}$.
\begin{example}[Denoising with bounded variation.]
\	
Let be $u, v \in L^2(\Omega)$. And let be $g: \Omega\rightarrow \mathbb{R}^n$, such that, $g\in C_0^\infty(\Omega, \mathbb{R}^n)$. Consider the following functional $J: L^2(\Omega) \rightarrow \mathbb{R}$, defined as follows,
	\begin{equation*}
		J(u)=\frac{1}{2} \int_{\Omega}\abs{u(x)-v(x)}^2+\alpha \sup_{\norm{g}\leq 1} \int_{\Omega} u \dive (g) dx
	\end{equation*}
	Also consider the minimization problem \[\min_{u\in BV(\Omega)} J(u),\] 
	restricted to the set of functions with bounded total variations,
	\[BV(\Omega)=\left\lbrace u\in L^1(\Omega) \ | \ V(u, \Omega) < \infty\right\rbrace,\]
	
	where a total bounded variation is defined as,
	\begin{equation*}
		V(u, \Omega)=\sup\left\lbrace 
		 \int_{\Omega} u\ \dive (g) dx;  \ \text{such that} \ g \in C_0^\infty (\Omega, \mathbb{R}^n), \  \norm{g}_\infty \leq 1
		\right\rbrace
	\end{equation*}

	 \begin{remark}
	 For $u$ smooth enough, it is possible to apply integration by parts, considering the contributions due $g$ has compact support and $\Omega\subset \mathbb{R}^n$, $\int_{\Omega} u\ \dive g dx = -\int_{\Omega} g\cdot \nabla u  dx$.
	 \end{remark}

	Consider the norm defined on $BV(\Omega)$ as follows,	
		\[\norm{u}_{BV}:= \norm{u}_{L^1(\Omega)}+V(u, \Omega).\]
	If we consider $J(u)=F(u)+G(Au)$, we can set
	\begin{align*}
		F(u)&=\frac{1}{2}\int_{\Omega} \abs{u(x)-v(x)}^2 dx = \frac{1}{2}\norm{u-v}_{L^2(\Omega)}^2 \\
		G(Au)&=\alpha\int_{\Omega} \abs{\nabla u}dx
	\end{align*}
	 Where $A:=\alpha \nabla$, and $G(u)=\int_{\Omega} \abs{u}dx$. We introduce the convex functional of each function,
	\begin{align*}
		F^*(q^*)&=\frac{1}{2} \int_{\Omega}\abs{q^*(x)-v(x)} ^2 -\frac{1}{2} v^2(x) dx & \quad \forall q^*\in L^2(\Omega)
\\
G^*(p^*) &={\left\lbrace 
	\begin{array}{cl}
		0, &\quad \norm{p^*} \leq 1 \\
	-\infty, &\quad \text{otherwise}
	\end{array}
	\right.}  & \quad \forall p^* \in C_0^\infty(\Omega, \mathbb{R}^n)
	\end{align*}
	
	In order to apply the Fencel duality we see that the , adjoint of $A$ is given by $A^\star=-\alpha (\nabla\cdot)$, thus	
	\[-J(p^*)=\frac{1}{2} \int_{\Omega} \abs{-\alpha \nabla\cdot p^* + v^2}^2+ \frac{1}{2} v^2 dx\]
\end{example}

\subsection{Lagrangians}
\begin{definition}
	The function $L:U\times Y^* \rightarrow \overline{\mathbb{R}}$, $-L(u, p^*)=\sup_{p\in Y} \left\lbrace p^*(p) - \Phi(u, p)
	\right\rbrace$, is called Lagrangian of \eqref{eq8. P} relative to the perturbation $\Phi$. If we denote by $\Phi_u$ for fixed $u\in U$ the function $p\rightarrow \Phi(u, p)$, then $-L(u, p^*)=\Phi^*_u(p^*)$
\end{definition}

\begin{lemma}
	For all $u\in U$, the function $L_u: Y^* \rightarrow \overline{\mathbb{R}}$, $p^*\rightarrow L(u, p)$ is  a concave function (i.e. $-L_u$ is convex) and weak upper semi-continuous. If $\Phi$ is convex then for all $p^* \in Y^*$ the function $L_{p^*}: U\rightarrow \overline{\mathbb{R}}$, $u\rightarrow L(u, p^*)$ is convex.
	\begin{proof}

	\end{proof}
\end{lemma}

Without assuming anything about $\Phi$, we obtain
\begin{align*}
	\Phi^*(u^*, p^*) &= \sup_{u\in U, p\in Y} \left\lbrace u^*(u) + p^*(p) - \Phi(u,p) \right\rbrace \\
	&= \sup_{u\in U} \left\lbrace u^*(u) + \sup_{p\in Y}\left[p^*(p) - \Phi(u,p)\right]
	\right\rbrace \\
	&= \sup_{u \in U}\left\lbrace  u^*(u)-L(u, p^*) \right\rbrace
\end{align*}
This implies that,
\[
	\eqref{eq8. P*} \sup_{p^* \in Y^*} \left\lbrace -\Phi^*(0, p^*) \right\rbrace = \sup_{p^* \in Y^*} \inf_{u \in U} L(u, p^*)
\]

Now we assume that $\Phi$ is convex and weak lower semi-continuous, then for $u\in U$, the function 
$\Phi_u : Y \rightarrow \overline{\mathbb{R}}$ is convex and weak lower semi-continuous and thus $\Phi_u^** =\Phi_u$. Moreover
\begin{align*}
	\Phi(u,p)&=\Phi_u^{**}(p) \\
	&=\sup_{p^* \in Y^*}\left\lbrace
	p^*(p)-\Phi_u^*(p)
	\right\rbrace\\ 
	&=\sup_{p^*\in Y^*}\left\lbrace
	p^*(p)+L(u, p^*)
	\right\rbrace \\
	&=\sup_{p^*\in Y^*}\left\lbrace L(u, p^*)\right\rbrace
\end{align*}

Thus,
\begin{equation}
\eqref{eq8. P} \quad \inf_{u, p} \Phi(u, p) = \inf_{u \in U} \sup_{p^*\in Y^*} L(u, p^*)
\end{equation}

\begin{remark}
	The problems \eqref{eq8. P} and \eqref{eq8. P*} are related to min-max problem we have that the weak duality means 
	\[
		\sup \inf L \leq \inf \sup L
	\]
\end{remark}

\begin{definition}
	An element $(\overline{u}, \overline{p^*}) \in U\times Y^*$ is called saddle point of $L$ if 
	\[
		L(\overline{u}, p^*) \leq L(\overline{u}, \overline{p^*}) \leq L(u, \overline{p^*}), \quad \forall u \in U, \forall p^* \in Y^*.
	\]
\end{definition}

\begin{theorem}
	Assume that $\Phi$ convex and weak lower semicontinuous. Then $(u^*, \overline{p^*})$ is a saddle point of $L$ if and only if $\overline{u}$ is solution of \eqref{eq8. P}, $\overline{p^*}$ is solution of \eqref{eq8. P*} and $\inf\eqref{eq8. P}=\sup\eqref{eq8. P*}$.
	\begin{proof}
		Let $(\overline{u}, \overline{p^*})$ be a saddle point of $L$. We have that, 
		\begin{align*}
		\left.
		\begin{array}{c}
		L(\overline{u}, \overline{p^*}) = \inf_{u \in U} L(u, \overline{p^*})=-\Phi^*(0, \overline{p^*}) \\
		L(\overline{u}, \overline{p^*}) = \sup_{p^* \in Y^*} L(\overline{u}, \overline{p^*})=-\Phi^*(\overline{u}, 0)
		\end{array}
		\right\rbrace \implies \Phi(\overline{u}, 0)+\Phi^*(0, \overline{p}^*)=0
		\end{align*}
		Theorem about extremal conditions $\implies$ $\overline{u}$ is a solution of $\eqref{eq8. P}$, $\overline{p^*}$ solution of $\eqref{eq8. P*}$ and 
		\[
			\inf\eqref{eq8. P} = \sup\eqref{eq8. P*}
		\]
		"other direction" follows the same argumentation.
	\end{proof}
\end{theorem}
\begin{theorem}[Saddle point theorem.]
	Let $\Phi: U \times Y \rightarrow \overline{\mathbb{R}}$ be convex, weak lower semicontinuous and $\eqref{eq8. P}$ is stable. Then $\overline{u}\in U$ is a solution of $\eqref{eq8. P}$ if and only if then exist $\overline{p^*} \in Y^*$ such that $(\overline{u}, \overline{p^*})$, is a saddle point of $L$.
\end{theorem}
