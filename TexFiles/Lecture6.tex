\begin{theorem}
	Let $V, U$, Banach Spaces. Let $J:V \rightarrow \overline{\mathbb{R}}$ a convex functional. And consider the mapping $A: U\rightarrow V$ linear and continuous with the adjoint $A^*:U^*\rightarrow V^*$. Moreover, $J$ is lower semi-continuous and let $A\overline{u}$ be a point where $J$ is continuous and finite. Then te compose function $J\circ A: U\rightarrow \overline{\mathbb{R}}$ is subdifferentiable for all $u\in V$ and, 
	\[
		\partial(J\circ A)(u)=A^*\left(\partial J(Au)\right)
	\]
	\begin{proof}
		Let $p^*\in \partial J(Au)$,
		\[
			J(p)\geq J(Au)+p^*(p-Au) \quad \forall p \in V 
		\]
		where $p=Av$ with $v\in U$,
		
		\begin{align*}
			(J\circ A) (v) &\geq (J\circ A) (u)+p^*(A(v-u)) \quad \forall v \in U \\
						   &=(J\circ A)(u)+A^*p^*(v-u) \quad \forall v \in U
		\end{align*}
		
		i.e. $A^*p^* \in \partial (J\circ A)(u) \implies A^* \partial J(Au)\subseteq \partial (J\circ A)(u)$. 
		
		Conversely, take $u^*\in \partial (J\circ A)(u)$.
		\begin{align*}
		\label{eq6. u* in subdifferential}
			\forall v \in V, \qquad u^*(v-u)+(J\circ A)(u)\leq (J\circ A)(v)  \tag{\EODeer}
		\end{align*}
		 
		
Let us consider the affine subspace in $V\times \mathbb{R}$.
\begin{equation*}
\mathscr{L}=\{\left(Av, u^*(v-u)+(J\circ A)(u) )\quad | \quad v \in V\right)\}
\end{equation*}

The inequality \eqref{eq6. u* in subdifferential}, shows that $\mathscr{L}$ and $\epi J$ only have boundary points in common. Since $J$ has been assumed to be convex and continuous at $A\overline{u}$, and $\interior{\epi J} \neq \emptyset$, by the Hahn-Banach theorem, there is a closed affine hyperplane $\mathscr{H}$ containing $\mathscr{L}$, which no intersects $\interior{\epi J}$. As usual, $\mathscr{H}$ is non-vertical, and therefore is the graph of a continuous affine function of $V$ into $\mathbb{R}$:
\begin{align*}
	p \mapsto p^*(p)+\alpha, \qquad \text{ where } p^* \in Y^*, \qquad \alpha \in \mathbb{R}.
\end{align*}
Since $\mathscr{H}$ contains $\mathscr{L}$:
\begin{align*}
	\forall v \in V, \qquad& p^*(p)+\alpha = u^*(v-u)+J\circ A(u)\\
	& \alpha = J\circ A(u)-u^*(u)\\
	\forall v \in V, \qquad & p^*(Av)=u^*(v)
\end{align*}
Thus  $u^*=A^*p^*$. Finally, since $\mathscr{H}$ does not intersect $\interior{\epi J}$:

\begin{align*}
	\forall p \in V, \qquad &p^*(p)+(J\circ A)(u) - A^*p^*(u) \leq J(p) \\
	& p^*(p-Au) + (J\circ A)(u)\leq J(p).
\end{align*}
Thus $p^* \in \partial F(Au)$. Whence the desired result: $u^*= A^*p^* \in A^*\partial J (Au)$, and
\begin{align*}
	\partial (J\circ A)(u)\subseteq A^* \partial J(Au).
\end{align*}
\end{proof}
\end{theorem}

\begin{theorem}
	If $J: U\rightarrow \overline{\mathbb{R}}$ is convex and Frech\'et-differentiable at $u\in U$, then $\partial J(u)=\lbrace J'(u)\rbrace$
	
	\begin{proof}
		Let $p^* \in \partial J(u)$. Then for each $t>0$, $J(u+tv)-J(u)\geq p^*(tv)=tp^*(v)$, diving by $t$ and taking the limit $t\rightarrow 0$ we obtain,
		\begin{align*}
			&J'(u;v) \geq p^*(v) \quad \forall v \in U \\
	\implies& (\nabla J(u)-p^*)(v) \geq 0 \quad \forall v \in U.
		\end{align*}
	Since $J(u)$ is Frech\'et differentiable the operator $J'(u;v)$ is linear with respect to $v$ and $p^* \in U^*$ implies $(\nabla J(u)-p^*)$ is linear, taking $-v \in U$, we obtain that $(\nabla J(u)-p^*)(v)\leq 0$. Therefore $p^*=\nabla J(u)$.
	
	On the other hand, if $J$ is differentiable, it follows that $\nabla J(u)\in \partial J(u)$. For $v\in U$, we set $w=v-u$, $u\in U$ we have by \eqref{prop5. First derivative convex.}, 
	\begin{align*}
		    J(u+w)-J(u)&\geq J'(u;w)=\left(w\mid \nabla J(u)\right)_{UU^*} \\
	\implies \ J(v)-J(u)&\geq J'(u; v-u) = \left(v-u \mid \nabla J(u)\right)_{UU^*}
	\end{align*}
	
	Since the above inequality holds for all $v \in U$ implies $\nabla J(u) \in \partial J(u)$.
	\end{proof}
\end{theorem}
\begin{remark}
	The subgradient can be used to obtain local optimality conditions that are necessary and sufficient for convex problem.
\end{remark}
\begin{theorem}
	\label{th6. 0 in subdifferential.}
	Let $U$ be a Banach Space and $J: U \rightarrow \mathbb{R}$ convex and proper. Then each local minimum is global minimum.
	Moreover $\overline{u}\in U$ is a minimizer if and only if $0 \in \partial J(\overline{u})$.
	\begin{proof}
		If $0 \in \partial J(\overline{u})$: $J(v) \geq J(\overline{u})+ (0)(v-\overline{u})=J(\overline{u})$, $\forall v\in U$, and hence $\overline{u}$ is a global minimizer. 
		
		
		Assume that $0 \notin J(\overline{u})$, then $\exists v \in U$, such that
		\[
			J(v) < J(\overline{u}) + (0)(v-\overline{u}) = J(\overline{u}).
		\]
		Therefore $\overline{u}$ cannot be a minimizer.
	\end{proof}
\end{theorem}

\begin{definition}[Duality]
Let $J:U\rightarrow \overline{\mathbb{R}}$, and $U$ a Banach space. Then the convex conjugate function $J^*: U^* \rightarrow \mathbb{R}$ is defined by
\[
	J^*(p^*)=\sup_{u\in U} \left\lbrace p^*(u)-J(u)\right\rbrace
\]
implies that $-\sup_{u\in U} \left\lbrace p^*(u)-J(u) \right\rbrace= -J^*(p^*)= \inf_{u\in U} \left\lbrace J(u)-p^*(u)\right\rbrace$.
\end{definition}

\begin{example}
	Consider the indicator function of a convex set $C$, $I_C: U\rightarrow\overline{\mathbb{R}}$
	\[
	I_C(u)=
	\left\lbrace
	\begin{array}{cl}
	0&\quad u \in C\\
	\infty & \quad \text{otherwise}.
	\end{array}
	\right.
	\]
	
	Then we have that the convex conjugate is given by
	\[J^*(p^*)=\sup_{u\in U}\left\lbrace p^*(u)-I_C(u)\right\rbrace=\sup_{u\in C}\left\lbrace p^*(u)\right\rbrace.\]
\end{example}
\begin{example}
	$J: \mathbb{R}_+\rightarrow \mathbb{R}$, such that
	\begin{equation*}
	J(u)=\left\lbrace
	\begin{array}{ll}
		u(\ln u -1) & \quad u >0 \\
		0 & \quad u =0
	\end{array}		
	\right.
	\end{equation*}
	Then $J^*(p^*)=\sup_{u>0} \{p^*u-u(\ln u -1)\}$, then for $u>0$,
	\begin{align*}
		f'(u)=p^*-(\ln u -1) -1 = p^* -\ln(u) \implies \overline{u}=\exp p^* \\
		f'(u)=p^*-(\ln u -1) -1 = p^* -\ln(u) \implies \overline{u}=\exp p^* \\
	\end{align*}
\end{example}
\begin{example}
	Let $J:\mathbb{R}\rightarrow \mathbb{R}$, such that $J(u)=\exp u$, then $J^*(p^*)=\sup_{u\in \mathbb{R}} \lbrace p^*u -\exp u\rbrace$. Let $f(u)=p^*u-\exp(u)$, therefore $f'(u)=p^*-\exp u$, $\forall u \in \mathbb{R}$. Which is zero for $\overline{u} = \ln p^*$, if $p^*>0$. Since $f''(u)<0$, then $\overline{u}$ is indeed maximum. And we see that $\lim\limits_{u\rightarrow \pm \infty} f(u)=-\infty$. If $p^*=0$, $f(u)=-\exp u < 0$ and therefore the $\sup_{u \in \mathbb{R}} f(u)=0$ (Consider the limit when $u \rightarrow -\infty$). Then we have,
	\begin{equation*}
		J^*(p^*)=\left\lbrace 
		\begin{array}{cr}
			p^*(\ln p^*-1) & \quad p^*>0 \\
			0 & \quad p^*=0
		\end{array}		
		\right.
	\end{equation*}	
\end{example}
\begin{example}
	Let $H$ be a Hilbert space and $J(u)=\frac{1}{2} \norm{u}^2$.
	Since $H$ is Hilbert, by Riesz, for each linear and bounded functional $\phi_{p^{*}} \in H$, $\exists p^* \in H$ such that, $\phi_{p^{*}}(u)=\langle u,p^*\rangle$. Using the definition of conjugate function,
	\begin{align*}
	J^*(p^*) &= \sup_{u \in U} \left\lbrace\langle u, p^*\rangle -\frac{1}{2} \norm{u}^2\right\rbrace \\
	&= -\inf_{u \in U} \left\lbrace \frac{1}{2} \norm{u}^2-\langle u, p^*\rangle \right\rbrace
	\end{align*}
Note that,
\[
	\frac{1}{2}\norm{u-p^*}^2=\frac{1}{2}\norm{u}^2-\langle u ,p^*\rangle + \frac{1}{2}\norm{p^*}^2 
\]
Therefore we can substitute in the above equation to find an equivalent form to the conjugate function,
\begin{align*}
J^*(p^*) &=-\inf_{u \in U} \left\lbrace \frac{1}{2} \left(\norm{u-p^*}^2 - \norm{p^*}^2\right) \right\rbrace \\
&=-\frac{1}{2} \inf_{u \in U} \left\lbrace  \norm{u-p^*}^2\right\rbrace +\frac{1}{2} \norm{p^*}^2
\end{align*}
We have $\norm{u-p^*} \geq 0$, $\forall u \in H$, then,
\[
\inf_{u \in U} \left\lbrace  \norm{u-p^*}^2\right\rbrace=0,
\]
since we can take $u=p^*$. Hence,
\begin{equation}
	J^*(p^*)=\frac{1}{2}\norm{p^*}^2
\end{equation}
\end{example}
\begin{theorem}
	Let $U$ be a Banach space and $J:U\rightarrow \overline{\mathbb{R}}$. Then $J^*$ is convex.
	\begin{proof}
		Let $p^*, q^* \in U^*$, and $\lambda \in [0,1]$,
		\begin{align*}
			J^*(\lambda p^* +(1-\lambda)q^*) &=\sup_{u\in U}\lbrace (\lambda p^* +(1-\lambda)q^*)(u)-J(u) \rbrace \\
										 &= \sup_{u \in U} \left\lbrace  \lambda p^*(u)	- \lambda J(u) +(1-\lambda)q^* (u)-(1-\lambda)J(u)\right\rbrace\\
										 &\leq \sup_{v,u\in U} \left\lbrace  \lambda p^*(u)	- \lambda J(u) +(1-\lambda)q^* (v)-(1-\lambda)J(v)\right\rbrace \\
										 &=\sup_{u\in U} \left\lbrace  \lambda p^*(u)	- \lambda J(u)\right\rbrace +\sup_{v \in U}\left\lbrace(1-\lambda)q^* (v)-(1-\lambda)J(v)\right\rbrace \\
										 &=\lambda J^*(p^*)+(1-\lambda) J^*(q^*).
		\end{align*}	
		Hence $J^*$ is convex.
	\end{proof}
\end{theorem}