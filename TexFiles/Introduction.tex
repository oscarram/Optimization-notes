


\subsection{Basic Definitions}

%----------------------------------------------------------
\begin{definition}[Notation for special sets.]
	\label{def0. Sets}
Let $X$ be a real  vector space, let $C$ and $D$ be subsets of $X$, and let $z \in X$. Then we use the following notation:
\begin{itemize}
	\item $C + D =\lbrace x + y \mid x \in C, y \in D\rbrace$.
	\item $C -D =\lbrace x - y \mid x \in C, y \in D\rbrace$.
	\item $z+C=\lbrace z \rbrace +C$.
	\item  $C-z=C-\lbrace z \rbrace$. 
	\item For every $\lambda \in \mathbb{R}$, $\lambda C =\lbrace\lambda x \mid x \in C\rbrace$.
	\item  If $\Lambda$ is a non-empty subset of $\mathbb{R}$, then $\Lambda C = \bigcup_{\lambda \in \Lambda}\lambda C$, and $\Lambda z = \Lambda \lbrace z \rbrace =\lbrace\lambda z \mid \lambda \in \Lambda$.
\end{itemize}
\end{definition}

 
%-------------------------------------------------------------------------------

\begin{definition}[Cone]
	Let $X$ be a vector space, and $C \subset X$. $C$ is called a cone with vertex $O$ if $C=\mathbb{R}^+C$. The cone is pointed or unpointed according to whether $O \in C$ or $\O \notin C$ respectively. A pointed cone with vertex  $O$ is salient if $C \cap \{-C\} = \{O\}$
\end{definition}

\begin{remark}[Ordering relation and Cones]
	We can associate a partial ordering relation denoted by $\preceq$(or $\succeq$) with a pointed cone $C$ by setting the following relation,
	\[
	 p \preceq q \iff q - p \in C.
	\]

	We can see the following properties, $p \preceq p$, $\forall p \in X$. If $p \preceq q $ and $q \preceq r$, then $p\preceq r$; the partial ordering relation is compatible with the structure of a vector space in the sense that
	\begin{align*}
		p \succeq 0 &\implies \lambda p \succeq 0, \quad \forall \lambda > 0 \\
		p \succeq q &\implies p+r \succeq q+r, \quad \forall r \in X		
	\end{align*}
	We see that the cone $C$ is the set of positive elements for this ordering relation,
	\[
		C = \{ p \in X \mid p \succeq 0\}
	\]
	The set $\{-C\}$ is the set of negative elements,
	\[
		\{-C\}=\{p \in X \mid p \preceq 0\}
	\]
	If the cone $C$ is salient, the relation $\preceq$ is an ordering relation:
	\[
		p\preceq q,\, q \preceq p \implies p = q
	\]
\end{remark}
\begin{definition}
	Let $C$ be a subset of $U$ Banach space. The conical hull of $C$ is the intersection of all the cones in $U$ containing $C$, i.e., the smallest cone in $U$ containing $C$. It is denoted by $\cone C$. The closed conical hull of $C$ is the smallest closed cone in $U$ containing $C$. It is denoted by $\overline{\cone} C.$
\end{definition}

\begin{definition}
	We say a functional $J$ is proper if $\dom J \neq \emptyset$ and $J>-\infty$.
\end{definition}

\begin{definition}
	Let $U$ a vector topological space. We define as the indicator function $I_C:U\rightarrow \overline{\mathbb{R}}$ of a set $C \subset U$, as follows:
	\begin{equation*}
		I_C(x)=\left\lbrace
		\begin{array}{ll}
		0	&\quad x \in C \\
		\infty &\quad \text{otherwise.}
		\end{array}
		\right.
	\end{equation*}
\end{definition}

\subsection{Useful lemmas and Theorems.}
\begin{lemma}
	\label{lemma0. Bounded and weakly convergent}
	Let $(x_n)_{n \in \mathbb{N}}$ be a bounded sequence in a Hilbert Space $H$. Then $(x_n)_{n \in \mathbb{N}}$ possesses a weakly convergent subsequence.
\end{lemma}

\begin{lemma}
Let $(x_n)_{n \in \mathbb{N}}$ be a sequence in a Hilbert Space $H$. Then $(x_n)_{n \in \mathbb{N}}$ converges if and only if it is bounded and possesses at most one weak sequential cluster point.
\end{lemma}
\begin{fact}
	A Banach space is reflexive if its unit ball is compact in the weak topology. This implies that every bounded sequence admits a weakly converging subsequence. Hilbert spaces and $L^p$ spaces ($1<p<\infty$) are reflexive.
\end{fact}
\begin{theorem} 
	Let $f: H \rightarrow (−\infty, \infty]$ be a convex functional on a Hilbert space. Then the following are equivalent:
	\begin{enumerate}[label=(\roman{*})]
\item 	$f$ is weakly sequentially lower semicontinuous.
\item 	$f$ is sequentially lower semicontinuous.
\item 	$f$ is lower semicontinuous.
\item 	$f$ is weakly lower semicontinuous.
	\end{enumerate}
\end{theorem}



\begin{lemma}
	A convex set is closed if and only if it is weakly closed.
\end{lemma}

\begin{lemma}
	Every bounded linear operator over a Banach Space is weakly continuos.
\end{lemma}

\begin{lemma}[Parallelogram law]
	
\[
\norm{x+y}^2 +\norm{x-y}^2 = 2\norm{x}^2+2\norm{y}^2
\]
\end{lemma}

\begin{lemma}
	Let $\mathcal{X}$ be a Hausdorff space and let $(f_i)_{i \in I}$ be a family of lower semicontinuous functions from $\mathcal{X}$ to $[-\infty, \infty]$. Then $\sup_{i\in I}f_i$ is lower semi-continuous. If $I$ is finite, then $\min_{i\in I}f_i$ is lower-semicontinuous.
\end{lemma}
\begin{definition}
	Let $\mathcal{X}$ be a Hausdorff space. The lower semicontinuous envelope of $f: \mathcal{X}\rightarrow[-\infty, \infty]$ is 
	\[
	\overline{f}=\sup \left\lbrace g: \mathcal{X}\rightarrow[-\infty, \infty] \ |\quad  g\leq f \text{ and } g \text{ is lower semicontinuous}\right\rbrace.
	\]
\end{definition}
\begin{proposition}
	If $C$ is a compact set in a normed space $U$, and $G$ is a closed subset of $C$. Then $G$ is compact.
	\begin{proof}
		Let $\{ g_n\}$ a sequence contained in $G$. Since $G \subset C$ and $C$ compact. $\exists \{g_n\}_k$ subsequence of $\{g_n\}$, contained in $G$ such that $\{g_n\}_k \rightarrow g$, as $k \rightarrow \infty$, and then since $G$ is closed $g \in G$. Therefore $G$ is compact.
	\end{proof}
	\label{prop0. closed subset compact. Precompact}
\end{proposition}

\begin{definition}
	Let $U$ a vector real space. We denote the set of all 
\end{definition}
\begin{proposition}
	\label{prop0. Convex and increasing derivative}
	Let $\phi:\mathbb{R}\rightarrow (-\infty, \infty]$ be a proper function that is differentiable on a nonempty open interval $C$ in $\dom \phi$. Then the following hold:
	\begin{itemize}
		\item $\phi'$ is increasing on $C$. Then $\phi + I_C$ is convex.
		\item Suppose that $\phi'$ is strictly increasing on $C$. Then $\phi$ is strictly convex on $C$. 
	\end{itemize}
\end{proposition}

%%%%% Text taken from Barbu%
% Fulfilling the gaps later.


%The locally convex topology defined
%by $\mathcal{P}$ on $X$ is called the weak topology or $Y-topology$ of X induced by the duality
%$(X,Y)$, and we denote it by $\sigma(X,Y)$. Similarly, the weak topology or X-topology
%of Y, denoted by $\sigma(Y, X)$, is the locally convex topology on Y generated by $Q$.
%Clearly, the roles of $X$ and $Y$ are interchangeable here, since there is a natural duality between $Y$ and $X$ which determines a dual system $(Y,X)$. Thus, it is sufficient to
%establish the properties only for the linear space $X$. According to the well-known re-sults concerning the locally convex topologies generated by families of seminorms,
%we immediately obtain the following result.
